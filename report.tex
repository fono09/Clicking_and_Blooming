\documentclass[10pt,a4j]{jreport}
\setlength{\oddsidemargin}{-1in}
\addtolength{\oddsidemargin}{10mm}
\setlength{\evensidemargin}{-1in}
\addtolength{\evensidemargin}{10mm}
\setlength{\textwidth}{\paperwidth}
\addtolength{\textwidth}{-20mm}

\begin{document}
\chapter{Clicking and Bloomingの実装について}
\section{背景}
限られた制作時間の中、簡易なゲームを素早く制作することに重点を置き、ユーザーが2D上の画面操作しているのにも関わらず、
3D描画処理をしていると実感できる描画を目指した。普通、3D空間を何らかの操作により移動可能にしたものが一般的に
3Dシューティングゲームと呼ばれるが、今回は一般的なタブレット端末でも遊べるようにし、UIは2D平面へのクリックとタップとした。
\section{設計}
プログラム上に作成したクラス(Title,Menu,Explosion,Sphere)は、Three.jsのオブジェクトをそのまま継承し仕様を簡素にした。
グローバルにゲーム全体の基本属性を置き、アクセスしやすさを確保した。3D描画性能が低い端末でも遊べるよう、
テクスチャや、ハイポリゴンを用いるのはなるべく回避した。
\section{実装}
RaycasterによりCameraからの光線トレースを行い、ゲーム内のオブジェクトが反応し、ほかのオブジェクトにもscene経由で作用するモデルとなっている。
あくまで、インスタンスのクリックイベント、タッチイベントにより駆動、また、リフレッシュ時に一定確率で遠くに球体が出現するようにした。
\subsection{球体出現}
まず、球体の出現をランダムで行うプロトタイプを書いた。光源は環境光のみで、赤色の床の平面と緑色の球体が飛んでくる動作を確認した。
次に、点光源の追加を行った。また、球体と地面のPhoneシェーディングを行うようにしたのに加え、球体をhsl色表現で明るい色のみとした。
\subsection{クリックの実装}
Raycasterによる、マウス操作との当たり判定を行い、球体を消去する動作を追加した。このとき、オブジェクトからカメラの距離を判定し、実際にクリックできていても、奥行き限界を設けた。
次に、ゲームの題名の元になる開花エフェクト(爆発エフェクト)この段階で実装を検討していたが、オブジェクティブな実装を行っていなかったため、
2度に渡るリファクタリングを実施する。
\subsection{リファクタリング}
リファクタリングを実施するに当たって、設計で述べたThree.jsのオブジェクトをなるべくそのまま継承し実装する方針を固め、
球体のみ、更新などを行うオブジェクトが存在していたソースコードを、ゲーム全体からの相互作用を行うモデルに転換する。
\subsection{爆発エフェクトの実装}
リファクタリング後、球体の削除と同時に、爆発エフェクトをコンストラクトする仕様を実装した。また、一般的なパーティクルから立方体状に拡散するサンプルコードから、
球体の中心から球体の表面に伸びるベクトルの仰角をランダムに変更することにより、花火のようなパーティクルの拡散が得られた。
\subsection{タイトルと得点システムの実装}
ほかのオブジェクトと同様にクリック可能な文字列(Menu)と、クリック不可能なタイトル(Title)を実装した。このとき、ステージ切り替えで、常に振りまいている球体を演出として爆発するようにした。
\subsection{チューニング}
数名の友人に触って貰い、最終調整を行ったが、現在すでに提出日の19:46となっており、これ以上の機能追加を断念した。
\section{評価}
以下に友人からの容赦ない評価を記す。
\subsection{機能要望1}
爆発エフェクトで飛び散ったパーティクルがほかの球体と当たり判定を行い、連鎖的に球体が爆発する演出を行う希望が出たが、新規にクリックイベント以外の当たり判定を行う必要があり、断念した。
\subsection{機能要望2}
テクスチャなどの演出があればよかった。たとえば球体に「単位」などというテクスチャや、球体に限らず意外な形のオブジェクトを飛んでくる等の演出があれば良かった。
\subsection{機能要望3}
爆発エフェクトなのだから、ガラスが割れるような音か、花火のような音があればよかった。また、立体的に点音源を移動できるライブラリがあるのでそれを利用すれば良かった。
\section{まとめ}
以上のように、実際のゲーム製品をよく遊んでいる人などからの機能要望にあるように、インタラクティブ性を演出するための要素技術として、より高度なテクスチャ描画や音響技術が、
スマートフォンなどのレベルでふんだんに用いられているので、今後ともこのような描画技術は発展すると思われる。
\end{document}
